% Change the size of the paper your CV will be printed on
% by entering either a4paper or letterpaper here.
\documentclass[a4paper]{ReadableCV}

% Set color of body text
\color{black}

\begin{document}
	
	% Set page colour using X11names colour definitions
	\setPageColour{white}
	
	% Set header details being aligned to the right or left
	% If an image is displayed it will be shown on the
	% opposite side to what is set here.
	\setHeaderAlignment{left}
	
	% Set colour of all headings, header highlights
	\setHeadingColours{SlateGray3}
	
	% Set image file to be displayed in header
	% If left blank no image is displayed
	%\setImage{profilepic.jpg}
	
	% If image not being displayed then user can
	% move contact details to opposite side of
	% page to name and jobtitle.
	% Use either opposite or below
	\setContactLocation{opposite}
	
	% Set up information needed for header.
	% If you do not want to include certain
	% information use {} instead. 
	\setYourName{Stefanos Stefanou}
	\setYourJobTitle{Software Engineer}
	\setYourMobileNo{+44(0)7706066959}
	%\setYourHomeNo{TODO}
	\setYourEmailAddr{stefannstefanou@gmail.com}
	\setYourWebAddr{http://stefanos.isactually.cool}
	
	% Display header information 
	\showHeader
	
	% Set up whether section headings are on the left or right
	\setSectionAlignment{left}
	
	% Creates a new section title / heading
	\newHeading{Personal Profile}
	
	I am a Software engineer based in Hemsworth, UK. I am enthusiastic about everything I do, I behave well under pressure, and I am passionate about excellence. I can acquire and apply knowledge fast, and I adapt quickly to any situation. I have over three years of commercial functional scala experience in various firms, working in data-intense event-sourced and distributed systems. My hobbies include interest in Mathematics, Physics and Functional programming.
	
	\newHeading{Core Skills}
	
	% Add up to nine core skills. If they are
	% all not needed use {} instead.
	\addSkills{Typelevel Scala}
	{Fs2}
	{Kafka}
	{Akka}
	{MongoDB,Elastic,Solr}
	{PostgresSQL}
	{Neo4J}
	{Docker}
	{AWS}
	
	\newHeading{Career Summary}
	
	% Set up whether job title or company printed first
	% Either use JobFirst or CompanyFirst
	\setJobCompanyOrder{JobFirst}
	
	% This displays the whole of the role information
	% including dates [1], job title [2],
	% company name [3] and role summary [4]
	% If a full history is required use \newrole and \roleAchievements
	% If only a brief description needed then just use \newrole
	\newRole{6/2021 - present}
	{Software Engineer}
	{GB Group Plc}
	{As a software engineer at GBG, i am working on the 'Investigate' Product, providing identity identification, location inteligence and fraud prevention on B2B clients around the world}
	
	\newRole{5/2020 - 6/2021}
	{Software Engineer}
	{J Sainsburys PLC - Sainsburys Tech}
	{Supporting the unified pricing and promotions system  of the Sainsburys Group, a event-sourced data-intensive distributed system, Extensive use of Kafka,Akka,Fs2 as well as various Typelevel libraries }
	
	\newRole{2/2019 - 5/2020}
	{Ascosiate Software Engineer}
	{TALOS Software LTD}
	{As an ascosiate software engineer at TALOS, i worked extensively on the companys biggest product yet, 'Simplybook', mainly around the 
	backend system. Use of Scala play, Typelevel Scala, Akka as well as Cypress/Gatling for testing purposes }
	
	
	\newRole{10/2018 - 2/2019}
	{Research Assistant}
	{University of Reading}
	{Developing customized filesystem solutions for High- performance computers using FUSE(ESIWACE Project), extensive use of C, Python, and non relational databases such as MongoDB and ElasticSearch}
	
	% move, duplicate, delete or comment out the following line if necessary
	\newpage
	
	\newHeading{Education}
	
	\newRole{2018-2022}
	{Bsc Computer Science}
	{University of Reading}
	{\begin{itemize}
			\item First Class Honours
			\item Dissertation: An AI-assisted decision making system for thyroid nodule classification 
		\end{itemize}
	}
	
	% This is training you have done in your own time
	\newHeading{Dinstictions}
	
	\newRole{2018}
	{AIIA Low Latency Video Streaming Challenge 2018 (1rst)}
	{Aristotle University of Thessaloniki, Greece}
	%{MULTIDRONE/ICARUS/AIIA Research Lab : Low Latency Video Streaming Challenge 2018 }
	{Developing a customized solution for ICARUS LAB experimental surveillance drones, transfering real-time video and telemetry data with minimal latency, the end-result, alanStreamer was written using C++, C, QT, gstream}
	
	\newRole{2016}
	{International Open Bidding Competition 2016 (Team 2nd)}
	{Organiser : Université Sorbonne Paris Nord, France}
	{Functional prototype of a web information system while meeting all requirements set by a project committee,Extensive use of PHP, HTML/CSS,  Node.js and MySQL}
	
	\newRole{2015}
	{28th National Information Technology Competition (30th place,Finals)}
	{Organiser : National Technical University of Athens}
	{Greece's national algorithmic student competition, finishing at 30th place (final stage), nation-wide, Various problems
	focused on Dynamic Programming, Minimum Spanning Trees and various number-theory related algorithms}
	
	

	
	% School education
	\newHeading{Interesting Projects}
	
	\newCourse{2016}{\href{https://github.com/noReasonException/DummyFileSystem}{DFS} }{\small{Dummy File System, C and Linux Kernel 4.0.X}}{}
	\newCourse{2016}{\href{https://github.com/noReasonException/SScheduler}{SScheduler} }{\small{Stefano's Scheduler, simple linux scheduler implementation for kernel 4.X}}{}
	\newCourse{2018}{\href{https://www.bookworld.gr/gr/book/bkid/234405/java-gia-ligous}{'Java for the few'(Book)} }{\small{Worked on 3 chapters, Book is used on 2 universities (Uniwa,AUEB) }}{}
	
	
	
	
	
	
	\clearpage
	
	
	
\end{document}